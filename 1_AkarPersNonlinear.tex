\documentclass[xcolor={dvipsnames}, 9pt]{beamer}
\usetheme{default}
\usepackage{amsmath, amsfonts, tikz, xcolor}
\usepackage[backend=bibtex]{biblatex}
\usepackage[cal = esstix]{mathalpha}
\usefonttheme{serif}

\addbibresource{referensi.bib}

\setbeamercolor{background canvas}{bg=blue!10!White}
\renewcommand{\emph}[1]{\textcolor{Blue}{#1}}

\title{MatNum: Akar Persamaan Tak Linear}
\author{Fadhlannafis K. K.}
\date{10122040}
\begin{document}
	\begin{frame}[plain]
		\maketitle
	\end{frame}
	\begin{frame}{Daftar Isi}
		\tableofcontents
	\end{frame}
	\begin{frame}{Catatan}
		PPT ini saya buat untuk menunjang pembelajaran matematika numerik saya, setelah UTS saya di bawah setengahnya rata-rata (alias kuartil 4). \newline
		Pembahasan di PPT ini cenderung teoritik. Saya mengambil referensi dari PPT Bu Lena dan dari:
		\nocite{*}
		\printbibliography
	\end{frame}
	\section{Masalah Penaksiran Akar}
	\subsection{Prinsip Kontraksi Banach}
	\begin{frame}{Prinsip Kontraksi Banach}
		Secara umum, sulit untuk menghitung akar dari persamaan tak linear. Sebagai contoh, akar dari persamaan $x^2y = \sin x$ sangat sulit untuk ditentukan. \newline
		Untuk sekarang, kita batasi pembahasan pada pendekatan akar dari persamaan $y=f(x)$, dengan $f\in C[a,b]$. \newline
		Untuk itu, kita akan membangun barisan yang konvergen ke akar dari persamaan tersebut. Tetapi, sekarang kita belum tahu apakah barisan tersebut akan konvergen atau tidak. Sebelum itu, kita perlu mengenal apa itu pemetaan kontraksi.
	\end{frame}
	\begin{frame}{Prinsip Kontraksi Banach}
		Misalkan $(X,d)$ ruang metrik, dan $f:X\to X$. Kita sebut $a\in X$ \emph{titik tetap} $f$ jika $f(a)=a$. Lalu, kita sebut $f$ sebagai \emph{kontraksi tegas} jika ada $\alpha\in[0,1)$ (disebut \emph{konstanta kontraksi/Lipschitz}) sehingga
		\begin{align}\label{DefKont}
			d(f(x),f(y))\leq\alpha d(x,y)
		\end{align}
		untuk setiap $x,y\in X$. \newline
		Perhatikan bahwa suatu fungsi $f$ bersifat kontraksi mengakibatkan ia kontinu. \newline 
		Nanti, kita akan banyak menggunakan pemetaan kontraksi untuk mencari barisan yang konvergen ke suatu akar persamaan nonlinear. Untuk itu, kita memerlukan suatu teorema.
		\begin{theorem}[Prinsip Kontraksi Banach]
			Setiap kontraksi tegas di suatu ruang metrik lengkap memiliki titik tetap tunggal.
		\end{theorem}
	\end{frame}
	\begin{frame}{Prinsip Kontraksi Banach}
		Bukti. Mengikuti \cite{Kainth2023Comprehensive}, misalkan $X$ ruang metrik lengkap dan $f:X\to X$ kontraksi tegas, dengan konstanta Lipschitz $\alpha$. \newline
		\boxed{\text{Ketunggalan.}} Jika $x,y$ adalah titik tetap berbeda dari $f$, maka
		\begin{align*}
			d(x,y) = d(f(x),f(y)) \leq \alpha d(x,y) < d(x,y),
		\end{align*}
		kontradiksi. \newline
		\boxed{\text{Eksistensi.}} Misalkan $x_0\in X$. Tulis $f^n(x)=f\circ f\circ f\circ \cdots \circ f(x)$, dengan komposisi dilakukan sebanyak $n$ kali. Tinjau barisan $\{f^n(x_0)\}$. Dengan menggunakan induksi pada \eqref{DefKont}, didapat
		\begin{align*}
			d(f^{n+1}(x_0),f^n(x_0))\leq\alpha d(f^n(x_0),f^{n-1}(x_0))\leq\cdots\leq\alpha^nd(f(x_0),x_0)=\alpha^nD_0,
		\end{align*}
		dengan $D_0=d(f(x_0),x_0)$. 
	\end{frame}
	\begin{frame}{Prinsip Kontraksi Banach}
		Misalkan $\varepsilon>0$. Karena $\alpha\in[0,1)$, deret $\sum_{n=1}^{\infty}\alpha^n$ konvergen. Dari kriteria Cauchy, ada suatu $m\in\mathbb{N}$ sehingga
		\begin{align*}
			\sum_{n=n_1}^{n_2-1}\alpha^n < \frac{\varepsilon}{1+D_0}
		\end{align*}
		untuk setiap $n_2>n_1>m$. Akibatnya, dari pertidaksamaan segitiga,
		\begin{align}\label{BCPpro1}
			d(f^{n_2}(x_0),f^{n_1}(x_0))\leq\sum_{n=n_1}^{n_2-1}d(f^{n+1}(x_0),f^n(x_0))\leq D_0\sum_{n=n_1}^{n_2-1}\alpha^n < \varepsilon.
		\end{align}
		Maka $\{f^n(x_0)\}$ Cauchy. Karena $X$ lengkap, $\{f^n(x_0)\}$ konvergen. Misalkan $x\in X$ sehingga $f^n(x_0)\to x$. Dari \eqref{DefKont}, $f$ kontinu di $x$. Karena $f^n(x_0)\to x$, didapat $f(f^n(x_0))\to f(x)$. Karena nilai limit tunggal, didapat
		\begin{align*}
			f(x) = \lim_{n\to\infty}f^{n+1}(x_0) = \lim_{n\to\infty}f^n(x_0) = x.
		\end{align*}
		Maka, $x$ adalah titik tetap bagi $f$.\qed\newline
		Perhatikan bahwa $x_0$ diambil sembarang.
	\end{frame}
	\subsection{Galat}
	\begin{frame}{Galat}
		Tentunya, karena kita akan membangun suatu metode numerik, pasti ada galat. Beberapa metode numerik akan dibangun dengan menggunakan Prinsip Kontraksi Banach. Sebagai kelanjutan dari teorema tersebut, kita punya teorema berikut.
		\begin{theorem}
			Misalkan $(X,d)$ ruang metrik lengkap, $f:X\to X$ kontraksi tegas dengan konstanta kontraksi $\alpha$ dan titik tetap $u$, dan $\{u_n\}_{n=0}^\infty\subset X$ dengan $u_{n+1} = f(u_n)$. Maka $u_n\to u$, dan batas-batas atas berikut berlaku:
			\begin{align}
				d(u_n,u)\leq\frac{\alpha^n}{1-\alpha}d(u_0,u_1)\label{Erbo1} \\
				d(u_n,u)\leq\frac{\alpha}{1-\alpha}d(u_{n-1},u_n)\label{Erbo2} \\
				d(u_n,u)\leq\alpha d(u_{n-1},u_n)\label{Erbo3}
			\end{align}
		\end{theorem}
	\end{frame}
	\begin{frame}{Galat}
		Bukti. Mengikuti \cite{Han2009Theoretical}, dari sifat kontraktif $f$, didapat
		\begin{align*}
			d(u_{n+1},u_n)\leq\alpha d(u_n,u_{n-1})\leq\cdots\leq\alpha^nd(u_1,u_0).
		\end{align*}
		Maka untuk sebarang $m\leq n\leq 1$,
		\begin{align}
			d(u_m,u_n)&\leq\sum_{j=0}^{m-n-1}d(u_{n+j+1},u_{n+j}) \nonumber\\
			&\leq\sum_{j=0}^{m-n-1}\alpha^{n+j}d(u_1,u_0)\leq\frac{\alpha^n}{1-\alpha}d(u_1,u_0)\label{errpr1}.
		\end{align}
		Karena $\alpha\in[0,1)$, didapat $d(u_m,u_n)\to 0$ ketika $n\to\infty$. Maka $\{u_n\}$ adalah barisan Cauchy. Karena $V$ lengkap, $\{u_n\}$ konvergen ke suatu titik $u'\in V$. Dengan mengambil $n\to\infty$ di persamaan $u_{n+1}=f(u_n)$, didapat $u'=u$ dari kontinuitas $f$.
	\end{frame}
	\begin{frame}{Galat}
		Dengan mengambil $m\to\infty$ di persamaan \eqref{errpr1}, didapat batas \eqref{Erbo1}. \newline
		Dari
		\begin{align*}
			d(u_n,u) = d(T(u_{n-1}),T(u)) \leq \alpha d(u_{n-1},u),
		\end{align*}
		kita dapatkan batas \eqref{Erbo3}. Dengan batas ini dan pertaksamaan segitiga, kita peroleh
		\begin{align*}
			d(u_n,u)&\leq \alpha d(u_{n-1},u)\leq \alpha(d(u_{n-1},u_n)+d(u_n,u))\\
			&=\alpha d(u_{n-1},u_n)+\alpha d(u_n,u)
		\end{align*}
		yaitu batas \eqref{Erbo2}, karena $\alpha\in[0,1)$.\qed
	\end{frame}
	\section{Metode Penaksiran Akar}
	\subsection{Akar Persamaan Nonlinear}
	\begin{frame}{Akar Persamaan Nonlinear}
		Sekarang misalkan $f:\mathbb{R}\to\mathbb{R}$. Kita akan mencari solusi dari persamaan
		\begin{align*}
			f(x) = 0, \, x\in\mathbb{R}.
		\end{align*}
		Untuk menggunakan Prinsip Kontraksi Banach, kita tulis ulang persamaan tersebut sebagai
		\begin{align}\label{operatorform}
			x=T(x), \, x\in\mathbb{R}
		\end{align}
		dengan $T:\mathbb{R}\to\mathbb{R}$ suatu fungsi yang bergantung pada $f$. Ingat bahwa $\mathbb{R}$ lengkap, maka ada tepat satu solusi dari persamaan tersebut. Sekarang hanya perlu menentukan $T$ yang memudahkan komputasi. \newline
		Dalam pencarian akar persamaan, ada dua tipe metode numerik:
		\begin{enumerate}
			\item \emph{metode pengurung} (akar yang dicari diapit interval),
			\item \emph{metode terbuka} (akar yang dicari tidak diapit interval).
		\end{enumerate} 
	\end{frame}
	\subsection{Metode-Metode Pengurung}
	\subsubsection{Metode Bagi Dua}
	\begin{frame}{Metode Pengurung: Bagi Dua}
		Metode ini tidak memerlukan Prinisp Kontraksi Banach. Ingat lagi Teorema Nilai Antara:
		\begin{theorem}
			Misalkan $f:\mathbb{R}\to\mathbb{R}$ kontinu di $[a,b]$, dan tulis $f(a) = \alpha, f(b) = \beta$. Maka untuk setiap $\gamma\in[\alpha,\beta]$, ada suatu $c\in [a,b]$ sehingga $f(c) = \gamma$.
		\end{theorem}
		Bukti. Tanya dosen Analisis Real kalian. \qed \newline
		Sekarang misalkan $f:[a,b]\to\mathbb{R}$ kontinu dengan $f(a)f(b)<0$ (artinya kedua nilai berbeda tanda). Dari TNA, pasti ada $c\in[a,b]$ sehingga $f(c)=0$.
	\end{frame}
	\begin{frame}{Metode Pengurung: Bagi Dua}
		Sekarang definisikan barisan $\{a_n\},\{b_n\},\{c_n\}$ secara rekursif sebagai berikut:
		\begin{gather*}
			a_1 = a, b_1 = b, c_n = \frac{a_n+b_n}{2},\\
			a_n = \begin{cases}
				c_{n-1}, & \text{jika } f(b_{n-1})f(c_{n-1})<0, \\
				a_{n-1}, & \text{selain itu},
			\end{cases}\\
			b_n = \begin{cases}
				c_{n-1}, & \text{jika } f(a_{n-1})f(c_{n-1})<0, \\
				b_{n-1}, & \text{selain itu}.
			\end{cases}
		\end{gather*}
		Perhitungan barisan dihentikan jika $f(c_k)=0$ untuk suatu $k$. \newline
		Yang menjadi masalah adalah kita tidak tahu kekonvergenan $c_n$.
	\end{frame}
	\begin{frame}{Metode Pengurung: Bagi Dua}
		\begin{lemma}
			Misalkan $\{a_n\},\{b_n\},\{c_n\}$ didefinisikan seperti di slide sebelumnya, dan $c$ adalah akar bagi $f$ di $[a,b]$. Maka $c_n\to c$.
		\end{lemma}
		Bukti. Menggunakan induksi, didapat $c\in[a_n,b_n]$ untuk setiap $n$. Karena panjang interval dari $[a_n,b_n]$ memenuhi $b_n-a_n = 2^{1-n}(b-a)$, haruslah $\ell[a_n,b_n] \to 0$ ketika $n\to\infty$. Akibatnya, $c_n = \frac{b_n+a_n}{2} \to c$. \qed \newline
		Maka, metode ini memiliki batas galat $|c_n-c|\leq \frac{b_1-a_1}{2^n}$.
	\end{frame}
	\begin{frame}{Metode Pengurung: Bagi Dua}
		Sekarang kita bisa mendefinisikan suatu prosedur untuk menghampiri akar dari $f:[a,b]\to\mathbb{R}$ yang kontinu menggunakan metode bagi dua dengan toleransi galat $\varepsilon$:
		\begin{enumerate}
			\item Tulis $c = \frac{b+a}{2}$.
			\item Jika $f(c) = 0$, hentikan proses perhitungan. \newline
			Jika $f(a)f(c) < 0$, tetapkan $b = c$. \newline
			Jika $f(b)f(c) < 0$, tetapkan $a = c$.
			\item Ulangi langkah 1-2 sampai $|b-a|<\varepsilon$. Ambil $c$ terakhir sebagai taksiran akar dari $f$.
		\end{enumerate}
	\end{frame}
	\subsubsection{Metode Posisi Palsu}
	\begin{frame}{Metode Posisi Palsu}
		Dengan \textit{setting} yang sama dengna metode bagi dua, definisikan $T(x) = $
	\end{frame}
\end{document}
