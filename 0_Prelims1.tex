\documentclass[xcolor={dvipsnames}, 9pt]{beamer}
\usetheme{default}
\usepackage{amsmath, amsfonts, tikz, xcolor, biblatex}
\usepackage[cal = esstix]{mathalpha}
\usefonttheme{serif}

\addbibresource{referensi.bib}

\setbeamercolor{background canvas}{bg=BrickRed!25!White}
\newcommand{\emp}[1]{\textcolor{Blue}{#1}}

\title{MatNum: Pengantar}
\author{Fadhlannafis K. K.}
\date{10122040}
\begin{document}
	\begin{frame}[plain]
		\maketitle
	\end{frame}
	\begin{frame}{Catatan}
		PPT ini saya buat untuk menunjang pembelajaran matematika numerik saya, setelah UTS saya di bawah setengahnya rata-rata (alias kuartil 4). \newline
		Pembahasan di PPT ini cenderung teoritik. Saya mengambil referensi dari PPT Bu Lena dan dari:
		\nocite{*}
		\printbibliography
	\end{frame}
	\begin{frame}{Apa Itu Numerik?}
		\emp{Matematika Numerik} adalah ilmu yang mempelajari prosedur untuk memecahkan masalah matematika menggunakan perhitungan yang sederhana. \newline
		\emp{Analisis Numerik} adalah matematika numerik, dengan mencakup pembuktian rumus-rumus yang diberikan. \newline
		\emp{Metode Numerik} adalah matematika numerik, dengan lebih menekankan pada penggunaan rumus. \newline
        Dalam mata kuliah Matematika Numerik, lebih ditekankan metode numerik. \newline
        Dalam slides ini, sangat ditekankan analisis numerik, dengan sedikit cara penggunaan.
	\end{frame}
	\begin{frame}{Tahapan}
		content...
	\end{frame}
	\begin{frame}{Algoritma}
		content...
	\end{frame}
	\begin{frame}{Galat}
		content...
	\end{frame}
	\begin{frame}{Angka Signifikan}
		content...
	\end{frame}
	\begin{frame}{Perlunya Analisis}
		Pada numerik, kita akan sering berhadapan dengan masalah \emp{konvergensi} dari metode yang kita pilih. \newline
		Nantinya, kita akan memperhatikan apakah barisan:
		\begin{enumerate}
			\item nilai hampiran akar,
			\item fungsi hampiran,
			\item vektor hampiran vektor eigen,
			\item dan lain sebagainya,
		\end{enumerate}
		itu konvergen ke yang sebenarnya. Untuk itu, diperlukan \emp{Analisis Fungsional}.
	\end{frame}
    \begin{frame}{Ruang Bernorm}
        Misalkan $V$ adalah ruang vektor atas lapangan $\mathbb{F}$. Kita sebut $V$ sebagai \emp{ruang bernorm} jika ada fungsi \emp{norm} $\|\cdot\|:V\to [0,\infty)$ sehingga untuk setiap $u,v\in V$ dan $\alpha\in\mathbb{F}$:
        \begin{enumerate}
            \item $\|v\|\geq 0$, dan $\|v\|=0$ jhj $v=0$,
            \item $\|\alpha v\|=|\alpha|\|v\|$,
            \item $\|u+v\|\leq\|u\|+\|v\|$.
        \end{enumerate}
        Suatu \emp{seminorm} adalah fungsi $\|\cdot\|:V\to[0,\infty)$ yang memenuhi ketiga sifat tadi, kecuali $\|v\|=0$ tidak menjamin $v=0$.
    \end{frame}
    \begin{frame}{Ruang Bernorm}
        Beberapa contoh ruang bernorm adalah:
        \begin{enumerate}
            \item Di $\mathbb{R}^d$, \emp{norm Euclid} didefinisikan sebagai
            \begin{align*}
                \|x\|_2 = \left(\sum_{i=1}^d x_i^2\right)^{1/2}.
            \end{align*}
            \item 
        \end{enumerate}
    \end{frame}
	\begin{frame}{Ruang Metrik}
		Misalkan $V$ adalah ruang vektor atas lapangan $\mathbb{F}$. Kita sebut $V$ sebagai \emp{ruang metrik} jika ada fungsi \emp{metrik} $d:V\times V\to [0,\infty)$ sehingga untuk setiap $u,v,w\in V$:
		\begin{enumerate}
			\item $d(v,w) \geq 0$, dan $d(v,w) = 0$ jhj $v=w$,
			\item $d(v,w) = d(w,v)$,
			\item $d(v,w) \leq d(v,u) + d(u,w)$.
		\end{enumerate}
        Terlihat bahwa ruang metrik adalah perumuman dari ruang bernorm.
	\end{frame}
    \begin{frame}{Ruang Metrik}
        
    \end{frame}
    \begin{frame}{Konvergensi}
        
    \end{frame}
\end{document}