\documentclass[xcolor={dvipsnames}, 9pt]{beamer}
\usetheme{default}
\usepackage{amsmath, amsfonts, tikz, xcolor, graphicx}
\usepackage[backend=bibtex]{biblatex}
\usepackage[cal = esstix]{mathalpha}
\usepackage[utf8]{inputenc}
\usepackage[T1]{fontenc} 
\usefonttheme{serif}

\addbibresource{referensimnb.bib}

\setbeamercolor{background canvas}{bg=blue!10!white}
\renewcommand{\emph}[1]{\textcolor{Blue}{#1}}

%\includeonlyframes{current}

\title{MA3171 Matematika Numerik \\ \textbf{Masalah Nilai Batas}}
\author{Fadhlannafis K. K., Gema Nadiku Pantouw, Muhammad Ariq Fakhri, Muhammad Arif Wibisono}
\date{10122040, 10122066, 10122076, 10122108}
\begin{document}
	\begin{frame}[plain]
		\maketitle
	\end{frame}
	\begin{frame}{Catatan}
		%PPT ini saya buat untuk menunjang pembelajaran matematika numerik saya, setelah UTS saya di bawah setengahnya rata-rata (alias kuartil 4). \newline
		Pembahasan di PPT ini cenderung teoritik. Kami mengambil referensi dari PPT Bu Lena dan dari:
		\nocite{*}
		\printbibliography
		Pustaka utama yang digunakan dalam slides ini adalah \cite{Ascher1995Numerical}.
	\end{frame}
	\begin{frame}{Pengantar}
		Suatu \emph{masalah nilai batas} adalah persamaan diferensial (dalam slides ini hanya yang biasa) dengan orde dua atau lebih, sehingga solusi $u$ memenuhi syarat batas
		\begin{align*}
			u(a) = A, u(b) = B.
		\end{align*}
		Sebagai contoh,
		\begin{align*}
			\begin{cases}
				u'' + p(x)u = f(x), & a<x<b, \\
				u(a) = A, u(b) = B.
			\end{cases}
		\end{align*}
		Untuk menyelesaikan masalah nilai batas secara numerik, ada dua metode:
		\begin{enumerate}
			\item Metode Tembakan/\textit{Shooting} Linear,
			\item Metode Beda Hingga.
		\end{enumerate}
	\end{frame}
	\begin{frame}{Pengantar}
		Untuk masalah persamaan diferensial, ada suatu teorema yang menjamin kekonvergenan solusi, yaitu
		\begin{theorem}[Ekuivalensi Lax]
			Misalkan suatu masalah persamaan diferensial terdefinisi dengan baik. Maka, suatu metode numerik konsisten dan stabil jika dan hanya jika dia konvergen.
		\end{theorem}
		Teorema tersebut tidak akan dibuktikan, karena sudah jauh di luar pembahasan slides ini.
	\end{frame}
	\begin{frame}{Metode Tembakan Linear: Pengantar}
		Kita tinjau masalah nilai batas linear:
		\begin{gather}
			u''(x) + p(x)u'(x) + q(x)u(x) = r(x), a<x<b, \label{mnb} \\
			u(a) = A, u(b) = B. \nonumber
		\end{gather}
		Asumsikan \eqref{mnb} memiliki solusi tunggal, yaitu $u(x)$. Ide dari metode tebakan linear adalah mendekati masalah \eqref{mnb} dengan suatu masalah nilai awal
		\begin{gather}
			u''(x) + p(x)u'(x) + q(x)u(x) = r(x), a<x<b, \label{mna} \\
			u(a) = A, u'(a) = A' \nonumber
		\end{gather}
		dengan $A'$ dipilih sedemikian sehingga $u(b)=B$. Setelah itu, kita bisa gunakan metode-metode numerik yang biasa digunakan untuk masalah nilai awal.
	\end{frame}
	\begin{frame}{Metode Tembakan Linear: Konstruksi}
		Yang menjadi masalah adalah bagaimana cara memilih $A'$. Kita tebak:
		\begin{align}
			u'(a) = s_1 \label{tebak1}\\
			u'(a) = s_2 \label{tebak2}
		\end{align}
		Masalah nilai awal \eqref{mna},\eqref{tebak1} dan \eqref{mna},\eqref{tebak2} dapat diselesaikan. Misalkan solusi dari \eqref{mna},\eqref{tebak1} adalah $v(x)$ dan solusi dari \eqref{mna},\eqref{tebak2} adalah $w(x)$. Kita bisa asumsikan $v(b)\neq B$, $w(b)\neq B$, dan $v\not\equiv w$. Maka $u\not\equiv v$, $u\not\equiv w$. Lalu, karena masalah \eqref{mna} linear, haruslah
		\begin{align*}
			u(x) = \theta w(x) + (1-\theta)v(x).
		\end{align*}
		untuk suatu $\theta\in\mathbb{R}$. \newline
		Sekarang, agar $u$ menjadi solusi \eqref{mnb}, haruslah
		\begin{align*}
			\theta w(b) + (1-\theta)v(b) &= B \\
			\Rightarrow \theta &= \frac{B - v(b)}{w(b)-v(b)}.
		\end{align*}
	\end{frame}
	\begin{frame}{Metode Tembakan Linear: Konstruksi}
		Dengan \textit{framework} sebelumnya, sekarang kita coba tinjau dua masalah nilai awal berikut:
		\begin{gather}
			u''(x) + p(x)u'(x) + q(x)u(x) = r(x), u(a) = A, u'(a) = 0 \label{mna2}\\
			u''(x) + p(x)u'(x) + q(x)u(x) = 0, u(a) = 0, u'(a) = B \label{mna3}
		\end{gather}
		Misalkan solusi dari \eqref{mna2} adalah $v_1$ dan dari \eqref{mna3} adalah $w_1$. Dari kelinearan, kita dapatkan solusi $u$ haruslah memenuhi
		\begin{align*}
			u(x) = v_1(x) + \theta w_1(x), \\
			u(a) = A, u(b) = v_1(b) + \theta w_1(b).
		\end{align*} 
		Maka
		\begin{align*}
			\theta = \frac{B - v_1(b)}{w_1(b)} \Rightarrow u(x) = v_1(x) + \frac{B - v_1(b)}{w_1(b)}.
		\end{align*}
		Metode ini disebut \emph{metode Shooting/tembakan}, karena seolah-olah kita menembakkan artileri di titik $x=a$ dari ketinggian $A$ sehingga "hadiah" yang kita tembakkan mengenai target di titik $x=b$ dengan ketinggian $B$.
	\end{frame}
	\begin{frame}{Metode Tembakan Linear: Galat}
		Dalam pembahasan ini, kita tinjau masalah nilai batas yang lebih umum:
		\begin{gather}
			\vec{y'} = A(x)\vec{y} + \vec{q(x)}, a<x<b \nonumber \\
			B_a\vec{y}(a) + B_b\vec{y}(b) = \vec{B} \label{genmnb}
		\end{gather}
		dengan mengingat bahwa masalah nilai batas orde-$n$ dapat diubah menjadi sistem persamaan diferensial orde 1. Maka solusi dari \eqref{genmnb} dapat ditulis sebagai
		\begin{align*}
			\vec{y}(x) = Y(x)\vec{s} + \vec{v}(x), a\leq x\leq b
		\end{align*}
		dengan $Y(x)$ adalah suatu solusi fundamental yang akan didefinisikan secara persis beberapa slide lagi, $\vec{s}$ adalah vektor parameter, dan $\vec{v}(x)$ adalah solusi partikular. Kita asumsikan
		\begin{align*}
			Y'(x) = A(x)Y, a<x<b, Y(a) = I.
		\end{align*} 
	\end{frame}
	\begin{frame}{Metode Tembakan Linear: Galat}
		Sekarang akan ditentukan galat dari metode tembakan. Tentunya galat dari metode ini sangat bergantung pada metode yang digunakan untuk menyelesaikan masalah nilai awal. Tinjau masalah \eqref{genmnb}. \newline
		Misalkan $\tau$ adalah toleransi galat untuk penyelesaian masalah nilai awal. Cara paling mudah untuk meminimalisasi galat adalah dengan mengendalikan galat lokal di $\vec{v}(x)$ dan $Y(x)$ sesuai dengan $\tau$, dan menentukan efeknya pada $\vec{y}(x)$. \newline
		Asumsikan bahwa setiap solusi numerik dari MNA dicari menggunakan titik-titik
		\begin{align*}
			a = x_1 < x_2 < \cdots < x_N < x_{N+1} = b,
		\end{align*}
		dan setiap solusi numerik $\{\vec{y}_i\}$ memenuhi
		\begin{align}\label{discretescheme}
			\vec{y}_{i+1} = \Gamma_i\vec{y}_i + \vec{g}_i, i = 1,\dots, N
		\end{align}
		dengan matriks $\Gamma_i$ dan vektor $\vec{g}_i$ ditentukan dari metode numerik yang digunakan untuk menyelesaikan MNA.
	\end{frame}
	\begin{frame}{Metode Tembakan Linear: Galat}
		Untuk melakukan analisis galat, diperlukan beberapa definisi:
		\begin{definition}
			\begin{enumerate}
				\item Kita definisikan $Y(x,t)$ sebagai suatu fungsi yang memenuhi $Y'(x,t) = A(x)Y(x,t)$ untuk $a<x<b$ dengan turunan terhadap $x$, $Y(t,t) = I$, dan $Y(x) \cong Y(x,a)$ invertibel untuk setiap $x$. Definisikan \emph{fungsi Green} sebagai
				\begin{align*}
					G(x,t) = \begin{cases}
						Y(x)Q^{-1}B_aY(a)Y^{-1}(t) & t\leq x \\
						-Y(x)Q^{-1}B_bY(b)Y^{-1}(t) & t > x
					\end{cases}
				\end{align*}
				dengan
				\begin{align*}
					Q = B_aY(a) + B_bY(b).
				\end{align*}
				\item Definisikan
				\begin{align*}
					\kappa_1 = \|YQ^{-1}\|_\infty = \sup_{x\in [a,b]} \|Y(x)Q^{-1}\|, 
					\kappa_2 = \left\|\int_a^b G(x,t) dt\right\|_\infty
				\end{align*}
				Kita sebut suatu masalah nilai batas \emph{well-conditioned} jika $\kappa = \max\{\kappa_1,\kappa_2\}$ nilainya tidak terlalu besar (bergantung pada masalah)
			\end{enumerate}
		\end{definition}
	\end{frame}
	\begin{frame}{Metode Tembakan Linear: Galat}
		\begin{definition}
			Kita definisikan $\{Y_i\}$ sehingga $Y_1 = I, Y_{i+1} = \Gamma_iY_i$ sebagai \emph{solusi fundamental diskrit} dari \eqref{discretescheme}, dan \emph{fungsi Green diskrit} sebagai
			\begin{align*}
				G_{ij} = \begin{cases}
					Y_i(Q^h)^{-1}B_aY_iY_j^{-1}, & j \leq i \\
					-Y_i(Q^h)^{-1}B_bY_{N+1}Y_j^{-1}, & j > i
				\end{cases},
				1 \leq i, j \leq N,
			\end{align*}
			dengan
			\begin{align*}
				Q^h = B_a + B_bY_{N+1}.
			\end{align*}
		\end{definition}
		Maka solusi numerik bentuk MNA dari \eqref{genmnb}, selain memenuhi \eqref{discretescheme}, juga harus memenuhi
		\begin{align*}
			\vec{y}_i(x) = Y_i(x)(Q^h)^{-1}\vec{B} + \sum_{j=1}^{N} G_{ij}\vec{g}_j.
		\end{align*}
	\end{frame}
	\begin{frame}{Metode Tembakan Linear: Galat}
		Sekarang misalkan metode numerik yang digunakan untuk menyelesaikan MNA memunculkan galat pemotongan $\frac{\vec{t}_i}{x_{i+1}-x_i}$ di $x_i$. Maka
		\begin{align*}
			\vec{y}(x_{i+1}) - \vec{y}_{i+1} = \Gamma_i(\vec{y}(x_i)-\vec{y}_i) + \vec{t}_i.
		\end{align*}
		Akibatnya
		\begin{align*}
			\vec{y}(x_i) - \vec{y}_i = \sum_{j=1}^N G_{ij}\vec{t}_j.
		\end{align*}
		Persamaan ini menghubungkan galat di $\vec{y}_i$ dengan galat dari MNA. \newline
		Dengan beberapa asumsi yang wajar, ada suatu konstanta $C$ sehingga
		\begin{align*}
			\|G_{ij}\| \leq C \|G(x_i,x_k)\|, 1\leq i,j \leq N.
		\end{align*}
	\end{frame}
	\begin{frame}{Metode Tembakan Linear: Galat}
		Jika kita tinjau galat per langkah, kita bisa tulis
		\begin{align*}
			\vec{t}_j = \tau \vec{c}_j(x_{j+1}-x_j), |\vec{c}_j| \approx 1.
		\end{align*}
		Akibatnya,
		\begin{gather*}
			\vec{y}(x_i) - \vec{y}_i = \tau \sum_{j=1}^{N}G_{ij}(x_{j+1}-x_j)\vec{c}_j \\
			\Rightarrow |\vec{y}(x_i) - \vec{y}_i| \leq K\tau \left\|\int_a^b G(x_i,t) dt\right\| = K\tau\kappa.
		\end{gather*}
		Maka secara umum metode tembakan linear akan memberikan galat diskritisasi jika \eqref{genmnb} well-conditioned.
	\end{frame}
	\begin{frame}{Metode Tembakan Linear: Kestabilan}
		Tentunya, kestabilan metode tembakan linear sangat bergantung pada metode yang digunakan untuk menyelesaikan masalah nilai awal. Asumsikan \eqref{genmnb} well-conditioned, dengan kata lain $\kappa$ kecil. \newline
		Misalkan $\vec{s}_h$ adalah pendekatan numerik dari $\vec{s}$ untuk MNB \eqref{genmnb}. Karena $Y(x)$ sudah didekati sebelumnya, kita dapat melihat perilaku $Y(x)(\vec{s}_h-\vec{s})$ untuk memperkirakan galat dari pendekatan $\vec{y}(x)$. Asumsikan ada matriks $E, \tilde{E}$ dengan norm kecil dan suatu vektor satuan $\vec{c}$, sehingga
		\begin{align*}
			(B_a+B_bY(b)(I+E))\vec{s}_h = \tilde{\vec{B}} = \hat{\vec{B}} - B_bY(b)\tilde{E}_c
		\end{align*}
		dengan
		\begin{gather*}
			\hat{\vec{B}} = \vec{B} - B_a\vec{v}(a) - B_b\vec{v}(b).
		\end{gather*}
		Dari menyelesaikan MNB, didapat $Q\vec{s} = \hat{\vec{B}}$. Maka
		\begin{align*}
			Y(x)(\vec{s}_h-\vec{s}) = -Y(x)Q^{-1}B_bY(b)(E\vec{s}_h+\tilde{E}\vec{c}).
		\end{align*}
		Karena kita asumsikan MNB well-conditioned, $\|YQ^{-1}\|$ tidak terlalu besar.
	\end{frame}
	\begin{frame}{Metode Tembakan Linear: Kestabilan}
		Tetapi, karena $E, \tilde{E}$ bisa sebarang matriks dengan norm kecil, bisa saja $Y(b)$ sangat memperbesar $E\vec{s}_h+\tilde{E}\vec{c}$, contohnya ketika $Y(x)$ memiliki elemen eksponensial. \newline
		Sekarang tinjau masalah \eqref{genmnb} dengan orde $n\geq 2$, dengan $n-v$ buah solusi yang meluruh dan $v\geq 1$ buah solusi yang naik, dan salah satunya naik lebih cepat, dan dengan syarat batas
		\begin{align*}
			B_a = \begin{bmatrix}
				B_{a1} \\ 0
			\end{bmatrix}, B_b = \begin{bmatrix}
				0 \\ B_{b2}
			\end{bmatrix}.
		\end{align*}
		Agar masalah seperti itu well-conditioned, haruslah
		\begin{align*}
			\text{rank}(B_{a1}) = n-v, \text{rank}(B_{b2}) = v.
		\end{align*}
		Jika ada salah satu solusi yang bertumbuh jauh lebih cepat dibandingkan sisanya di $x=b$, secara efektif $\text{rank}(Y(b)) = 1$, akibatnya $B_{b2}Y(b)$ juga rank 1, dan
		\begin{align*}
			\text{rank}(Q) = \text{rank}\left(\begin{bmatrix}
				B_{a1} \\ B_{b2}Y(b)
			\end{bmatrix}\right) \leq n-v+1.
		\end{align*}
		Akibatnya, jika $v>1$, secara efektif rank $Q$ kurang dari $n$, dan $\vec{s}_h$ tidak bisa dijamin akurat.
	\end{frame}
	\begin{frame}[fragile]{Metode Tembakan Linear: Kode Matlab}
		Kode Matlab dari metode tembakan linear bisa ditulis sebagai berikut:
		\begin{verbatim}
			function X = linshoot(a,b,n,p,q,r,xa,xb)
			% Pendefinisian variabel
			h = (b-a)/n;
			X = zeros(n+1,1);
			U = zeros(n+1,1);
			U(1) = xa;
			U1 = zeros(n+1,1);
			U1(1) = 0;
			U2 = zeros(n+1,1);
			V = zeros(n+1,1);
			V(1) = 0;
			V1 = zeros(n+1,1);
			V1(1) = 1;
			V2 = zeros(n+1,1);
		\end{verbatim}
	\end{frame}
	\begin{frame}[fragile]{Metode Tembakan Linear: Kode Matlab}
		\begin{verbatim}
			% Mencari solusi dari u(t)
			U2(1) = p(a)*U1(1) + q(a)*U(1) + r(a);
			for j = 2:1:(n+1)
			U(j) = U(j-1) + h*U1(j-1);
			U1(j) = U1(j-1) + h*U2(j-1);
			U2(j) = p(a+h*(j-1))*U1(1) + q(a+h*(j-1))*U(1) + r(a+h*(j-1));
			end
			% Mencari solusi dari v(t)
			V2(1) = p(a)*V1(1) + q(a)*V(1) + r(a);
			for j = 2:1:(n+1)
			V(j) = V(j-1) + h*V1(j-1);
			V1(j) = V1(j-1) + h*V2(j-1);
			V2(j) = p(a+h*(j-1))*V1(1) + q(a+h*(j-1))*V(1) + r(a+h*(j-1));
			end
			% Mencari solusi dari x(t)
			for j = 1:1:(n+1)
			X(j) = U(j) + ((xb - U(n+1))*V(j))/V(n+1);
			end
			end
		\end{verbatim}
		% JANGAN HAPUS [fragile]
	\end{frame}
	\begin{frame}[fragile]{Metode Tembakan Linear: Contoh}
		Sebagai contoh, tinjau masalah
		\begin{align*}
			x''(t) = tx'(t) - 2x(t) + 2 \\
			x(a) = 0, x(b) = 1.
		\end{align*}
		Di Matlab, tulis
		\begin{verbatim}
			a = 0; % Batas bawah t
			b = 1; % Batas atas t
			n = 10;% Banyaknya partisi
			% x(t) = t^2
			p = @(t) t;
			q = @(t) -2;
			r = @(t) 2;
			xa = 0; % Nilai x(a)\\
			xb = 1; % Nilai x(b)
			X2 = linshoot(a,b,n,p,q,r,xa,xb);\\
			T = linspace(a,b,n+1);
			plot(T,X2)
			title('Visualisasi Hasil Metode Shooting Linier')
			xlabel('t')
			ylabel('x')
		\end{verbatim}
	\end{frame}
	\begin{frame}{Metode Tembakan Linear: Contoh}
		Dari kode tersebut, didapat grafik:
		\begin{center}
			\includegraphics[scale=0.4]{linearshooting.pdf}
		\end{center}
	\end{frame}
	\begin{frame}{Metode Beda Hingga: Perbandingan Dengan Metode Tembakan Linear}
		Metode tembakan linear yang dijelaskan sebelumnya hanya melakukan \emph{satu kali} penembakan. Ternyata, bisa ditunjukkan bahwa metode beda hingga itu ekuivalen dengan suatu metode tembakan berulang kali, yang ada di luar pembahasan slides ini. Akibatnya, metode beda hingga akan dijamin stabil melalui
		\begin{theorem}
			Misalkan suatu skema beda hingga memenuhi
			\begin{align*}
				|\Psi(\vec{u},\vec{v};x,h)| \leq c(|\vec{u}|+|\vec{v}|)
			\end{align*}
			dan MNB \eqref{genmnb} terdefinisi dengan baik. Maka skema beda hingga tersebut stabil.
		\end{theorem}
		Seperti yang dilihat di slides-slides sebelumnya, untuk metode tembakan linear \emph{tidak ada} jaminan seperti ini. \newline
		Bukti dari teorema tersebut memerlukan pembahasan yang lebih dalam tentang metode beda hingga. Buktinya ada di \cite{Ascher1995Numerical}, halaman 201-202.
	\end{frame}
	\begin{frame}{Contoh Permasalahan}
		
		Sebuah pembangkit listrik menggunakan sistem pendingin berbasis pipa untuk membuang panas dari ruang mesin ke lingkungan. Panas tersebut dipindahkan melalui pipa panjang \( L = 10 \, \text{m} \) yang dikelilingi oleh udara dengan suhu lingkungan \( T_\infty = 25^\circ \, \text{C} \). Sepanjang pipa, panas hilang ke lingkungan melalui mekanisme konveksi. Suhu pipa \( T(x) \) sepanjang \( x \) (jarak dari ujung pertama pipa) dapat dimodelkan dengan persamaan diferensial berikut:
		
		\[
		\frac{d^2T(x)}{dx^2} + h \cdot \frac{P}{A} \cdot (T(x) - T_\infty) = 0
		\]
		
		\begin{enumerate}
			\item \( T(x) \): Suhu pipa pada posisi \( x \) (°C),
			\item \( x \): Jarak dari ujung pertama pipa (m),
			\item \( h = 50 \, \text{W/m}^2\text{K} \): Koefisien perpindahan panas konveksi,
			\item \( P = 0.2 \, \text{m} \): Keliling pipa (m),
			\item \( A = 0.01 \, \text{m}^2 \): Luas penampang pipa (m²),
			\item \( T_\infty = 25^\circ \, \text{C} \): Suhu udara luar.
			
		\end{enumerate}
	\end{frame}
	
	\begin{frame}{Contoh Permasalahan}
		\[
		\frac{d^2T(x)}{dx^2} + h \cdot \frac{P}{A} \cdot (T(x) - T_\infty) = 0
		\]
		Diberitahukan sebuah kondisi batas di $x=0$, Suhu awal pipa diketahui, \( T(0) = 150^\circ \, \text{C} \). Di $x=L=10$ dan panas pada ujung pipa terdisipasi sempurna, sehingga gradien suhu \( \frac{dT}{dx} = 0 \).
		\vspace{1 cm}\\
		\textbf{Permasalahan}
		\begin{enumerate}
			\item Tentukan distribusi suhu $T(x)$ di sepanjang pipa kemudian suhu pada $x=5$m dan $x=L$ serta serta gambarkan grafik distribusi suhu $T(x)$
			\item Pastikan bahwa suhu maksimum pipa tidak melebihi $T_{max} = 300^{\circ} C$. Jika suhu melebihi batas tersebut, diskusikan bagaimana mengubah parameter $h, P, A$ agar suhu tetap aman
		\end{enumerate}
		
	\end{frame}
	
	\begin{frame}{Pembahasan}
		\begin{figure}[h] 
			\centering
			\includegraphics[width=0.7\textwidth]{Pipa.png}
		\end{figure}
		Melalui fungsi Matlab dengan metode Beda Hingga dan Shooting Linear, didapatkan grafik di atas berikut. Terlihat perbedaan antara hasil metode \textit{shooting linear} dengan metode beda hingga.
		
	\end{frame}
	
	\begin{frame}{Pembahasan}
		\begin{figure}[h] 
			\centering
			\includegraphics[width=0.7\textwidth]{Pipa.png}
		\end{figure}
		Hasil luaran matlab pun memberikan luaran suhu pipa di 5 m yaitu $95.0199^{\circ}C$ untuk metode
		(\textit{shooting linear}) dan $-25^{\circ}$C untuk metode (\textit{beda hingga}) 
		dan untuk suhu pipa di ujung (10 m) yaitu $0^{\circ}$C untuk kedua metode.
	\end{frame}
	
	%	\begin{frame}{Metode Beda Hingga: Pengantar}
		%		kemungkinan salah total, ntar gw bikin t e lax di sini. \newline
		%		Pada dasarnya, metode beda hingga ingin mendekati solusi analitik dengan polinom piecewise. Untuk membangun metode beda hingga, di slides ini akan digunakan \emph{prinsip Rayleigh-Ritz} dan \emph{prinsip Galerkin}. Sebelum itu, diperlukan beberapa definisi. \par
		%		\begin{definition}[Ruang $L^2$]
			%			Misalkan $a,b\in\mathbb{R}, a<b$. \emph{Ruang $L^2(a,b)$} didefinisikan sebagai himpunan semua fungsi $v:[a,b]\to\mathbb{R}$ dengan \emph{hasil kali dalam} yang didefinisikan sebagai
			%			\begin{align*}
				%				\langle u,v\rangle_{L^2(a,b)} = \left(\int_{a}^{b} |u(x)v(x)| dx\right)^{1/2} < \infty.
				%			\end{align*}
			%		\end{definition}
		%	\end{frame}
	%	\begin{frame}{Metode Beda Hingga: Pengantar}
		%		\begin{definition}[Ruang Sobolev]
			%			Misalkan $k\in\mathbb{N}$, dan $a,b\in\mathbb{R}, a<b$. Kita definisikan \emph{Ruang Sobolev} $H^k(a,b)$ sebagai himpunan semua fungsi $v:[a,b]\to\mathbb{R}$ sehingga $v$ dan turunan-turunannya sampai orde ke $k-1$ kontinu absolut di $[a,b]$ dan $v^{(k)}\in L^2(a,b)$, yang dilengkapi hasil kali dalam
			%			\begin{align*}
				%				\langle u,v\rangle_{H^k(a,b)} = \left(\sum_{m=0}^{k}\langle u^{(m)},v^{(m)}\rangle_{L^2(a,b)}\right)^{1/2}.
				%			\end{align*}
			%		\end{definition}
		%		\begin{definition}[Ruang Hilbert]
			%			Misalkan $(V,\langle\cdot,\cdot\rangle)$ adalah ruang hasil kali dalam terhadap lapangan $\mathbb{F}$. Kita sebut $V$ sebagai \emph{ruang Hilbert} jika terhadap norm $\|v\|=\langle v,v\rangle$, $V$ lengkap, dengan kata lain setiap barisan Cauchy di $V$ konvergen di $V$.
			%		\end{definition}
		%		\begin{definition}[Ruang Dual]
			%			Misalkan $V$ adalah ruang vektor terhadap lapangan $\mathbb{F}$. \emph{Ruang dual} dari $V$, ditulis $V'$, didefinisikan sebagai ruang vektor berisi semua transformasi linear $l:V\to\mathbb{F}$, dengan $l$ disebut \emph{fungsional linear}, dengan operasi jumlah dan perkalian skalar di $\mathbb{F}$.
			%		\end{definition}
		%	\end{frame}
	%	\begin{frame}{Metode Beda Hingga: Konstruksi}
		%		Di sini, pertama akan ditinjau konstruksi metode Galerkin pada ruang Hilbert $V$, mengikuti \cite{Han2009Theoretical}. Bisa dibuktikan bahwa ruang $L^2$ dan ruang Sobolev adalah ruang Hilbert. Maka, konstruksi yang ada di sini adalah perumuman dari metode beda hingga. \par
		%		Misalkan $V$ adalah ruang Hilbert, $a(\cdot,\cdot):V\times V\to \mathbb{R}$ adalah fungsi yang linear terhadap kedua argumennya (disebut \emph{bilinear}), dan $l\in V'$. Kita akan tinjau masalah
		%		\begin{align}\label{fdm_cons1}
			%			u\in V, \, a(u,v) = l(v) \, \forall v\in V,
			%		\end{align}
		%		dengan asumsi $a(\cdot,\cdot)$ terbatas dan $V$-eliptik
		%		\begin{align}
			%			|a(u,v)|&\leq M\|u\|_V\|v\|_V \, \forall u,v\in V, \label{fdm_cons2} \\
			%			a(v,v) &\geq c_0\|v\|_V^2 \, \forall v\in V. \label{fdm_cons3}
			%		\end{align}
		%		untuk suatu $M,c_0>0$.  
		%	\end{frame}
	%	\begin{frame}{Metode Beda Hingga: Konstruksi}
		%		Teorema berikut menjamin bahwa masalah \eqref{fdm_cons1} memiliki solusi tunggal. 
		%		\begin{theorem}[Lemma Lax-Milgram]
			%			Misalkan $V$ adalah ruang Hilbert, $a(\cdot,\cdot)$ adalah form bilinear terbatas dan $V$-eliptik di $V$, dan $l\in V'$. Maka ada solusi tunggal dari masalah
			%			\begin{align}\label{laxmilgram}
				%				u\in V, a(u,v)=l(v) \, \forall v\in V.
				%			\end{align}
			%		\end{theorem}
		%		Bukti. Memerlukan beberapa teorema yang sudah sangat jauh di luar konteks slide ini. Tetapi, idenya adalah menyadari bahwa \eqref{laxmilgram} ekuivalen dengan suatu masalah nilai tetap, lalu gunakan teorema-teorema yang sangat jauh tersebut. $\square$ \newline
		%		Tapi, secara umum, sulit menyelesaikan masalah tersebut karena $V$ biasanya berdimensi tak hingga.
		%	\end{frame}
	%	\begin{frame}{Metode Beda Hingga: Konstruksi}
		%		Dengan asumsi \eqref{fdm_cons2}, \eqref{fdm_cons3}, kita batasi masalah \eqref{fdm_cons1} ke subruang $V_n\subset V$ yang berdimensi $n$, yaitu
		%		\begin{align}\label{fdm_cons4}
			%			u_n\in V_n, \, a(u_n, v) = l(v), \, \forall v\in V_n.
			%		\end{align}
		%		Karena $V_n$ berdimensi hingga, kita bisa definisikan basis $\{\phi_i\}_{i=1}^n$ sehingga
		%		\begin{align*}
			%			u_n = \sum_{j=1}^{n} \xi_j\phi_j.
			%		\end{align*}
		%		Sekarang dengan menuliskan $v$ sebagai $\phi_i$ di \eqref{fdm_cons4} untuk setiap $i$, kita dapatkan SPL
		%		\begin{align}
			%			A\vec{\xi} = \vec{b}
			%		\end{align}
		%		dengan $A = [a(\phi_j,\phi_i)]$, $\vec{\xi} = [\xi_j]$, dan $\vec{b} = [l(\phi_i)]$. Maka \eqref{fdm_cons4} bisa diselesaikan dengan cara menyelesaikan SPL seperti di ALE.
		%	\end{frame}
	%	\begin{frame}{Metode Beda Hingga: Konstruksi}
		%		Tentunya solusi $u_n$ secara umum akan berbeda dari $u$. Agar solusi lebih akurat, bisa ditentukan solusi $u_N$ di $V_N\subset V$ sehingga $N>n$. Maka, untuk suatu barisan subruang $V_{n_1} \subset V_{n_2} \subset \cdots \subset V$, kita bisa hitung suatu barisan solusi $u_{n_i}\in V_{n_i}$. Prosedur ini disebut \emph{metode Galerkin}. \newline
		%		Sekarang kita akan tinjau masalah \eqref{fdm_cons1} di ruang Sobolev. Sebelum itu, perlu dikenalkan
		%        \begin{definition}
			%            \begin{enumerate}
				%                \item Misalkan $A,B\in\mathbb{R}$. Kita definisikan $H^1_E(a,b)$ sebagai himpunan semua fungsi $v\in H^1(a,b)$ sehingga $v(a)=A, v(b)=B$.
				%                \item Kita definisikan $H^1_0(a,b)$ sebagai himpunan semua fungsi $v\in H^1(a,b)$ sehingga $v(a)=v(b)=0$.
				%            \end{enumerate}
			%        \end{definition}
		%	\end{frame}
	%    \begin{frame}{Metode Beda Hingga: Konstruksi}
		%        Kita tinjau masalah nilai batas
		%        \begin{gather}
			%            -\frac{d}{dx}\left(p(x)\frac{du}{dx}\right) + q(x)\frac{du}{dx} + r(x)u = f(x), a<x<b, \label{mnb3} \\
			%            u(a) = A, u(b) = B \nonumber \\
			%            p\in C^1[a,b], r\in C[a,b], f\in L^2[a,b], p(x)\geq c_0>0, r(x)\geq 0. \nonumber
			%        \end{gather}
		%        Untuk menyelesaikan masalah tersebut, kita definisikan
		%        \begin{align}\label{weakform}
			%            \mathcal{I}(u) = \frac{1}{2}\int_a^b (p(x)(u')^2+r(x)u^2) dx - \int_a^b f(x)u(x)dx
			%        \end{align}
		%        dengan $u\in H^1_E(a,b)$, dan kita tinjau masalah optimisasi
		%        \begin{align}\label{rayleighritz}
			%            \min_{u\in H^1_E(a,b)} \mathcal{I}(u).
			%        \end{align}
		%        yang disebut sebagai \emph{prinsip Rayleigh-Ritz}. Nanti akan dibuktikan bahwa fungsi yang menjadi solusi masalah tersebut juga menyelesaikan masalah nilai batas di awal.
		%    \end{frame}
	%    \begin{frame}{Metode Beda Hingga: Konstruksi}
		%       Definisikan suatu form bilinear simetrik di $H^1(a,b)$:
		%       \begin{align*}
			%           \mathcal{A}(w,v) = \int_a^b (p(x)w'(x)v'(x)+r(x)w(x)v(x)) dx.
			%       \end{align*}
		%       Akibatnya, kita bisa tulis \eqref{weakform} sebagai
		%       \begin{align}\label{weakform2}
			%           \mathcal{I}(w) = \frac{1}{2}\mathcal{A}(w,w) - \langle f,w\rangle_{L^2(a,b)}, \, w\in H^1_E(a,b).
			%       \end{align}
		%       Dengan formulasi seperti ini, kita punya teorema.
		%       \begin{theorem}[Ekuivalensi Prinsip Rayleigh-Ritz]
			%           Suatu fungsi $u\in H^1_E(a,b)$ meminimumkan $\mathcal{I}$ di $H^1_E(a,b)$ jika dan hanya jika $u$ memenuhi \emph{prinsip Galerkin}
			%           \begin{align}\label{galerkin}
				%               \mathcal{A}(u,v) = \langle f,v\rangle_{L^2(a,b)} \, \forall v\in H^1_0(a,b).
				%           \end{align}
			%       \end{theorem}
		%       Bukti. Dijamin dari Lemma Lax-Milgram. $\square$ \newline
		%       Tentunya mencari solusi untuk masalah \eqref{weakform} juga sulit. Maka, kita bisa gunakan metode Galerkin yang sudah didefinisikan sebelumnya.
		%    \end{frame}
	%    \begin{frame}{Metode Beda Hingga: Konstruksi}
		%        Kita batasi masalah nilai batas \eqref{mnb} pada subruang berdimensi-$n>2$ $S_E^n\subset H_E^1(a,b)$. Salah satu cara paling mudah adalah dengan mendefinisikan fungsi $\psi\in H_E^1(a,b)$, misalkan
		%		\begin{align*}
			%			\psi(x) = \frac{B-A}{b-a}(x-a)+A
			%		\end{align*}
		%		dan fungsi-fungsi $\phi_j\in H_0^1(a,b), j = 1,\dots,n-1$ yang bebas linear, dan mendefinisikan
		%		\begin{align*}
			%			S_E^n = \{v_n\in H_E^1(a,b) : v_n = \psi + \sum_{i=1}^{n-1}a_i\phi_i, a_i\in\mathbb{R}\}.
			%		\end{align*}
		%		Kita tinjau suatu aproksimasi dari masalah Rayleigh-Ritz:
		%		\begin{align*}
			%			\min_{w_n \in S_E^n} \mathcal{I}(w_n),
			%		\end{align*}
		%		dan tentunya kita punya teorema yang merupakan aproksimasi dari Prinsip Galerkin. Hal tersebut kita sebut sebagai \emph{metode Galerkin}.
		%    \end{frame}
	%    \begin{frame}{Metode Beda Hingga: Konstruksi}
		%    	Sekarang dari metode Galerkin, akan dikonstruksi metode beda hingga. 
		%    \end{frame}
\end{document}
