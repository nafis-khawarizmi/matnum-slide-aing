\documentclass[xcolor={dvipsnames}, 9pt]{beamer}
\usetheme{default}
\usepackage{amsmath, amsfonts, tikz, xcolor, biblatex}
\usepackage[cal = esstix]{mathalpha}
\usefonttheme{serif}

\addbibresource{referensi.bib}

\setbeamercolor{background canvas}{bg=BrickRed!25!White}
\renewcommand{\emph}[1]{\textcolor{Blue}{#1}}

\title{MatNum: Masalah Nilai Batas}
\author{Fadhlannafis K. K., Muhammad Ariq Fakhri, Muhammad Arif Wibisono}
\date{10122040, 10122076, 10122108}
\begin{document}
	\begin{frame}[plain]
		\maketitle
	\end{frame}
	\begin{frame}{Catatan}
		%PPT ini saya buat untuk menunjang pembelajaran matematika numerik saya, setelah UTS saya di bawah setengahnya rata-rata (alias kuartil 4). \newline
		Pembahasan di PPT ini cenderung teoritik. Kami mengambil referensi dari PPT Bu Lena dan dari:
		\nocite{*}
		\printbibliography
	\end{frame}
	\begin{frame}{Pengantar}
		Suatu \emph{masalah nilai batas} adalah persamaan diferensial (dalam slides ini hanya yang biasa) dengan orde dua atau lebih, sehingga solusi $u$ memenuhi syarat batas
		\begin{align*}
			u(a) = A, u(b) = B.
		\end{align*}
		Untuk menyelesaikan masalah nilai batas secara numerik, ada dua metode:
		\begin{enumerate}
			\item Metode Galerkin (khususnya metode beda hingga),
			\item Metode Tebakan/\textit{Shooting} Linear.
		\end{enumerate}
	\end{frame}
	\begin{frame}{Metode Beda Hingga: Pengantar}
		Pada dasarnya, metode beda hingga ingin mendekati solusi analitik dengan polinom piecewise. Untuk membangun metode beda hingga, di slides ini akan digunakan \emph{prinsip Galerkin}. Sebelum itu, diperlukan beberapa definisi. \\
		\begin{definition}[Ruang $L^2$]
			Misalkan $a,b\in\mathbb{R}, a<b$. \emph{Ruang $L^2(a,b)$} didefinisikan sebagai himpunan semua fungsi $v:[a,b]\to\mathbb{R}$ dengan \emph{norm} yang didefinisikan sebagai
			\begin{align*}
				\|v\|_2 = \|v\|_{L^2(a,b)} = \left(\int_{a}^{b} |v(x)|^2 dx\right)^{1/2} < \infty.
			\end{align*}
		\end{definition}
		\begin{definition}[Ruang Sobolev]
			Misalkan $k\in\mathbb{N}$, dan $a,b\in\mathbb{R}, a<b$. Kita definisikan \emph{Ruang Sobolev} $H^k(a,b)$ sebagai himpunan semua fungsi $v:[a,b]\to\mathbb{R}$ sehingga $v$ dan turunan-turunannya sampai orde ke $k-1$ kontinu absolut di $[a,b]$ dan $v^{(k)}\in L^2(a,b)$, dan norm
			\begin{align*}
				\|v\|_{H^k(a,b)} = \left(\sum_{m=0}^{k}\|v^{(m)}\|^2_{L^2(a,b)}\right)^{1/2}
			\end{align*}
		\end{definition}
	\end{frame}
\end{document}
