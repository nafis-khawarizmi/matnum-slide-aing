\documentclass[xcolor={dvipsnames}, 9pt]{beamer}
\usetheme{default}
\usepackage{amsmath, amsfonts, tikz, xcolor}
\usepackage[backend=bibtex]{biblatex}
\usepackage[cal = esstix]{mathalpha}
\usefonttheme{serif}

\addbibresource{referensi.bib}

\setbeamercolor{background canvas}{bg=BrickRed!25!White}
\renewcommand{\emph}[1]{\textcolor{Blue}{#1}}

\title{MatNum: Masalah Nilai Batas}
\author{Fadhlannafis K. K., Gema Nadiku Pantouw, Muhammad Ariq Fakhri, Muhammad Arif Wibisono}
\date{10122040, 10122066, 10122076, 10122108}
\begin{document}
	\begin{frame}[plain]
		\maketitle
	\end{frame}
	\begin{frame}{Catatan}
		%PPT ini saya buat untuk menunjang pembelajaran matematika numerik saya, setelah UTS saya di bawah setengahnya rata-rata (alias kuartil 4). \newline
		Pembahasan di PPT ini cenderung teoritik. Kami mengambil referensi dari PPT Bu Lena dan dari:
		\nocite{*}
		\printbibliography
	\end{frame}
	\begin{frame}{Pengantar}
		Suatu \emph{masalah nilai batas} adalah persamaan diferensial (dalam slides ini hanya yang biasa) dengan orde dua atau lebih, sehingga solusi $u$ memenuhi syarat batas
		\begin{align*}
			u(a) = A, u(b) = B.
		\end{align*}
		Sebagai contoh,
		\begin{align*}
			\begin{cases}
				u'' + p(x)u = f(x), & a<x<b, \\
				u(a) = A, u(b) = B.
			\end{cases}
		\end{align*}
		Untuk menyelesaikan masalah nilai batas secara numerik, ada dua metode:
		\begin{enumerate}
			\item Metode Tebakan/\textit{Shooting} Linear,
			\item Metode Galerkin (khususnya metode beda hingga).
		\end{enumerate}
	\end{frame}
	\begin{frame}{Metode Tebakan Linear}
		
	\end{frame}
	\begin{frame}{Metode Beda Hingga: Pengantar}
		kemungkinan salah total, ntar gw bikin t e lax di sini. \newline
		Pada dasarnya, metode beda hingga ingin mendekati solusi analitik dengan polinom piecewise. Untuk membangun metode beda hingga, di slides ini akan digunakan \emph{prinsip Rayleigh-Ritz} dan \emph{prinsip Galerkin}. Sebelum itu, diperlukan beberapa definisi. \par
		\begin{definition}[Ruang $L^2$]
			Misalkan $a,b\in\mathbb{R}, a<b$. \emph{Ruang $L^2(a,b)$} didefinisikan sebagai himpunan semua fungsi $v:[a,b]\to\mathbb{R}$ dengan \emph{hasil kali dalam} yang didefinisikan sebagai
			\begin{align*}
				\langle u,v\rangle_{L^2(a,b)} = \left(\int_{a}^{b} |u(x)v(x)| dx\right)^{1/2} < \infty.
			\end{align*}
		\end{definition}
	\end{frame}
	\begin{frame}{Metode Beda Hingga: Pengantar}
		\begin{definition}[Ruang Sobolev]
			Misalkan $k\in\mathbb{N}$, dan $a,b\in\mathbb{R}, a<b$. Kita definisikan \emph{Ruang Sobolev} $H^k(a,b)$ sebagai himpunan semua fungsi $v:[a,b]\to\mathbb{R}$ sehingga $v$ dan turunan-turunannya sampai orde ke $k-1$ kontinu absolut di $[a,b]$ dan $v^{(k)}\in L^2(a,b)$, yang dilengkapi hasil kali dalam
			\begin{align*}
				\langle u,v\rangle_{H^k(a,b)} = \left(\sum_{m=0}^{k}\langle u^{(m)},v^{(m)}\rangle_{L^2(a,b)}\right)^{1/2}.
			\end{align*}
		\end{definition}
		\begin{definition}[Ruang Hilbert]
			Misalkan $(V,\langle\cdot,\cdot\rangle)$ adalah ruang hasil kali dalam terhadap lapangan $\mathbb{F}$. Kita sebut $V$ sebagai \emph{ruang Hilbert} jika terhadap norm $\|v\|=\langle v,v\rangle$, $V$ lengkap, dengan kata lain setiap barisan Cauchy di $V$ konvergen di $V$.
		\end{definition}
		\begin{definition}[Ruang Dual]
			Misalkan $V$ adalah ruang vektor terhadap lapangan $\mathbb{F}$. \emph{Ruang dual} dari $V$, ditulis $V'$, didefinisikan sebagai ruang vektor berisi semua transformasi linear $l:V\to\mathbb{F}$, dengan $l$ disebut \emph{fungsional linear}, dengan operasi jumlah dan perkalian skalar di $\mathbb{F}$.
		\end{definition}
	\end{frame}
	\begin{frame}{Metode Beda Hingga: Konstruksi}
		Di sini, pertama akan ditinjau konstruksi metode Galerkin pada ruang Hilbert $V$, mengikuti \cite{Han2009Theoretical}. Bisa dibuktikan bahwa ruang $L^2$ dan ruang Sobolev adalah ruang Hilbert. Maka, konstruksi yang ada di sini adalah perumuman dari metode beda hingga. \par
		Misalkan $V$ adalah ruang Hilbert, $a(\cdot,\cdot):V\times V\to \mathbb{R}$ adalah fungsi yang linear terhadap kedua argumennya (disebut \emph{bilinear}), dan $l\in V'$. Kita akan tinjau masalah
		\begin{align}\label{fdm_cons1}
			u\in V, \, a(u,v) = l(v) \, \forall v\in V,
		\end{align}
		dengan asumsi $a(\cdot,\cdot)$ terbatas dan $V$-eliptik
		\begin{align}
			|a(u,v)|&\leq M\|u\|_V\|v\|_V \, \forall u,v\in V, \label{fdm_cons2} \\
			a(v,v) &\geq c_0\|v\|_V^2 \, \forall v\in V. \label{fdm_cons3}
		\end{align}
		untuk suatu $M,c_0>0$.  
	\end{frame}
	\begin{frame}{Metode Beda Hingga: Konstruksi}
		Teorema berikut menjamin bahwa masalah \eqref{fdm_cons1} memiliki solusi tunggal. 
		\begin{theorem}[Lemma Lax-Milgram]
			Misalkan $V$ adalah ruang Hilbert, $a(\cdot,\cdot)$ adalah form bilinear terbatas dan $V$-eliptik di $V$, dan $l\in V'$. Maka ada solusi tunggal dari masalah
			\begin{align}\label{laxmilgram}
				u\in V, a(u,v)=l(v) \, \forall v\in V.
			\end{align}
		\end{theorem}
		Bukti. Memerlukan beberapa teorema yang sudah sangat jauh di luar konteks slide ini. Tetapi, idenya adalah menyadari bahwa \eqref{laxmilgram} ekuivalen dengan suatu masalah nilai tetap, lalu gunakan teorema-teorema yang sangat jauh tersebut. $\square$ \newline
		Tapi, secara umum, sulit menyelesaikan masalah tersebut karena $V$ biasanya berdimensi tak hingga.
	\end{frame}
	\begin{frame}{Metode Beda Hingga: Konstruksi}
		Dengan asumsi \eqref{fdm_cons2}, \eqref{fdm_cons3}, kita batasi masalah \eqref{fdm_cons1} ke subruang $V_n\subset V$ yang berdimensi $n$, yaitu
		\begin{align}\label{fdm_cons4}
			u_n\in V_n, \, a(u_n, v) = l(v), \, \forall v\in V_n.
		\end{align}
		Karena $V_n$ berdimensi hingga, kita bisa definisikan basis $\{\phi_i\}_{i=1}^n$ sehingga
		\begin{align*}
			u_n = \sum_{j=1}^{n} \xi_j\phi_j.
		\end{align*}
		Sekarang dengan menuliskan $v$ sebagai $\phi_i$ di \eqref{fdm_cons4} untuk setiap $i$, kita dapatkan SPL
		\begin{align}
			A\vec{\xi} = \vec{b}
		\end{align}
		dengan $A = [a(\phi_j,\phi_i)]$, $\vec{\xi} = [\xi_j]$, dan $\vec{b} = [l(\phi_i)]$. Maka \eqref{fdm_cons4} bisa diselesaikan dengan cara menyelesaikan SPL seperti di ALE.
	\end{frame}
	\begin{frame}{Metode Beda Hingga: Konstruksi}
		Tentunya solusi $u_n$ secara umum akan berbeda dari $u$. Agar solusi lebih akurat, bisa ditentukan solusi $u_N$ di $V_N\subset V$ sehingga $N>n$. Maka, untuk suatu barisan subruang $V_{n_1} \subset V_{n_2} \subset \cdots \subset V$, kita bisa hitung suatu barisan solusi $u_{n_i}\in V_{n_i}$. Prosedur ini disebut \emph{metode Galerkin}. \newline
		Sekarang kita akan tinjau masalah \eqref{fdm_cons1} di ruang Sobolev. Sebelum itu, perlu dikenalkan
        \begin{definition}
            \begin{enumerate}
                \item Misalkan $A,B\in\mathbb{R}$. Kita definisikan $H^1_E(a,b)$ sebagai himpunan semua fungsi $v\in H^1(a,b)$ sehingga $v(a)=A, v(b)=B$.
                \item Kita definisikan $H^1_0(a,b)$ sebagai himpunan semua fungsi $v\in H^1(a,b)$ sehingga $v(a)=v(b)=0$.
            \end{enumerate}
        \end{definition}
	\end{frame}
    \begin{frame}{Metode Beda Hingga: Konstruksi}
        Kita tinjau masalah nilai batas
        \begin{gather}
            -\frac{d}{dx}\left(p(x)\frac{du}{dx}\right) + q(x)\frac{du}{dx} + r(x)u = f(x), a<x<b, \label{mnb} \\
            u(a) = A, u(b) = B \nonumber \\
            p\in C^1[a,b], r\in C[a,b], f\in L^2[a,b], p(x)\geq c_0>0, r(x)\geq 0. \nonumber
        \end{gather}
        Untuk menyelesaikan masalah tersebut, kita definisikan
        \begin{align}\label{weakform}
            \mathcal{I}(u) = \frac{1}{2}\int_a^b (p(x)(u')^2+r(x)u^2) dx - \int_a^b f(x)u(x)dx
        \end{align}
        dengan $u\in H^1_E(a,b)$, dan kita tinjau masalah optimisasi
        \begin{align}\label{rayleighritz}
            \min_{u\in H^1_E(a,b)} \mathcal{I}(u).
        \end{align}
        yang disebut sebagai \emph{prinsip Rayleigh-Ritz}. Nanti akan dibuktikan bahwa fungsi yang menjadi solusi masalah tersebut juga menyelesaikan masalah nilai batas di awal.
    \end{frame}
    \begin{frame}{Metode Beda Hingga: Konstruksi}
       Definisikan suatu form bilinear simetrik di $H^1(a,b)$:
       \begin{align*}
           \mathcal{A}(w,v) = \int_a^b (p(x)w'(x)v'(x)+r(x)w(x)v(x)) dx.
       \end{align*}
       Akibatnya, kita bisa tulis \eqref{weakform} sebagai
       \begin{align}\label{weakform2}
           \mathcal{I}(w) = \frac{1}{2}\mathcal{A}(w,w) - \langle f,w\rangle_{L^2(a,b)}, \, w\in H^1_E(a,b).
       \end{align}
       Dengan formulasi seperti ini, kita punya teorema.
       \begin{theorem}[Ekuivalensi Prinsip Rayleigh-Ritz]
           Suatu fungsi $u\in H^1_E(a,b)$ meminimumkan $\mathcal{I}$ di $H^1_E(a,b)$ jika dan hanya jika $u$ memenuhi \emph{prinsip Galerkin}
           \begin{align}\label{galerkin}
               \mathcal{A}(u,v) = \langle f,v\rangle_{L^2(a,b)} \, \forall v\in H^1_0(a,b).
           \end{align}
       \end{theorem}
       Bukti. Dijamin dari Lemma Lax-Milgram. $\square$ \newline
       Tentunya mencari solusi untuk masalah \eqref{weakform} juga sulit. Maka, kita bisa gunakan metode Galerkin yang sudah didefinisikan sebelumnya.
    \end{frame}
    \begin{frame}{Metode Beda Hingga: Konstruksi}
        Kita batasi masalah nilai batas \eqref{mnb} pada subruang berdimensi-$n>2$ $S_E^n\subset H_E^1(a,b)$. Salah satu cara paling mudah adalah dengan mendefinisikan fungsi $\psi\in H_E^1(a,b)$, misalkan
		\begin{align*}
			\psi(x) = \frac{B-A}{b-a}(x-a)+A
		\end{align*}
		dan fungsi-fungsi $\phi_j\in H_0^1(a,b), j = 1,\dots,n-1$ yang bebas linear, dan mendefinisikan
		\begin{align*}
			S_E^n = \{v_n\in H_E^1(a,b) : v_n = \psi + \sum_{i=1}^{n-1}a_i\phi_i, a_i\in\mathbb{R}\}.
		\end{align*}
		Kita tinjau suatu aproksimasi dari masalah Rayleigh-Ritz:
		\begin{align*}
			\min_{w_n \in S_E^n} \mathcal{I}(w_n),
		\end{align*}
		dan tentunya kita punya teorema yang merupakan aproksimasi dari Prinsip Galerkin. Hal tersebut kita sebut sebagai \emph{metode Galerkin}.
    \end{frame}
    \begin{frame}{Metode Beda Hingga: Konstruksi}
    	Sekarang dari metode Galerkin, akan dikonstruksi metode beda hingga. 
    \end{frame}
\end{document}
