\documentclass[xcolor={dvipsnames}, 9pt]{beamer}
\usetheme{default}
\usepackage{amsmath, amsfonts, tikz, xcolor, biblatex}
\usepackage[cal = esstix]{mathalpha}
\usefonttheme{serif}

\addbibresource{referensi.bib}

\setbeamercolor{background canvas}{bg=BrickRed!25!White}
\renewcommand{\emph}[1]{\textcolor{Blue}{#1}}

\title{MatNum: Masalah Nilai Batas}
\author{Fadhlannafis K. K., Muhammad Ariq Fakhri, Muhammad Arif Wibisono}
\date{10122040, 10122076, 10122108}
\begin{document}
	\begin{frame}[plain]
		\maketitle
	\end{frame}
	\begin{frame}{Catatan}
		%PPT ini saya buat untuk menunjang pembelajaran matematika numerik saya, setelah UTS saya di bawah setengahnya rata-rata (alias kuartil 4). \newline
		Pembahasan di PPT ini cenderung teoritik. Kami mengambil referensi dari PPT Bu Lena dan dari:
		\nocite{*}
		\printbibliography
	\end{frame}
	\begin{frame}{Pengantar}
		Suatu \emph{masalah nilai batas} adalah persamaan diferensial (dalam slides ini hanya yang biasa) dengan orde dua atau lebih, sehingga solusi $u$ memenuhi syarat batas
		\begin{align*}
			u(a) = A, u(b) = B.
		\end{align*}
		Sebagai contoh,
		\begin{align*}
			\begin{cases}
				u'' + p(x)u = f(x), & a<x<b, \\
				u(a) = A, u(b) = B.
			\end{cases}
		\end{align*}
		Untuk menyelesaikan masalah nilai batas secara numerik, ada dua metode:
		\begin{enumerate}
			\item Metode Galerkin (khususnya metode beda hingga),
			\item Metode Tebakan/\textit{Shooting} Linear.
		\end{enumerate}
	\end{frame}
	\begin{frame}{Metode Beda Hingga: Pengantar}
		Pada dasarnya, metode beda hingga ingin mendekati solusi analitik dengan polinom piecewise. Untuk membangun metode beda hingga, di slides ini akan digunakan \emph{prinsip Rayleigh-Ritz} dan \emph{prinsip Galerkin}. Sebelum itu, diperlukan beberapa definisi. \par
		\begin{definition}[Ruang $L^2$]
			Misalkan $a,b\in\mathbb{R}, a<b$. \emph{Ruang $L^2(a,b)$} didefinisikan sebagai himpunan semua fungsi $v:[a,b]\to\mathbb{R}$ dengan \emph{hasil kali dalam} yang didefinisikan sebagai
			\begin{align*}
				\langle u,v\rangle_{L^2(a,b)} = \left(\int_{a}^{b} |u(x)v(x)| dx\right)^{1/2} < \infty.
			\end{align*}
		\end{definition}
	\end{frame}
	\begin{frame}{Metode Beda Hingga: Pengantar}
		\begin{definition}[Ruang Sobolev]
			Misalkan $k\in\mathbb{N}$, dan $a,b\in\mathbb{R}, a<b$. Kita definisikan \emph{Ruang Sobolev} $H^k(a,b)$ sebagai himpunan semua fungsi $v:[a,b]\to\mathbb{R}$ sehingga $v$ dan turunan-turunannya sampai orde ke $k-1$ kontinu absolut di $[a,b]$ dan $v^{(k)}\in L^2(a,b)$, dan hasil kali dalam
			\begin{align*}
				\langle u,v\rangle_{H^k(a,b)} = \left(\sum_{m=0}^{k}\langle u^{(m)},v^{(m)}\rangle_{L^2(a,b)}\right)^{1/2}.
			\end{align*}
		\end{definition}
		\begin{definition}[Ruang Hilbert]
			Misalkan $(V,\langle\cdot,\cdot\rangle)$ adalah ruang hasil kali dalam terhadap lapangan $\mathbb{F}$. Kita sebut $V$ sebagai \emph{ruang Hilbert} jika terhadap norm $\|v\|=\langle v,v\rangle$, $V$ lengkap, dengan kata lain setiap barisan Cauchy di $V$ konvergen di $V$.
		\end{definition}
		\begin{definition}[Ruang Dual]
			Misalkan $V$ adalah ruang vektor terhadap lapangan $\mathbb{F}$. \emph{Ruang dual} dari $V$, ditulis $V'$, didefinisikan sebagai ruang vektor berisi semua transformasi linear $l:V\to\mathbb{F}$, dengan $l$ disebut \emph{fungsional linear}, dengan operasi jumlah dan perkalian skalar di $\mathbb{F}$.
		\end{definition}
	\end{frame}
	\begin{frame}{Metode Beda Hingga: Konstruksi}
		Di sini, akan ditinjau konstruksi metode Galerkin pada ruang Hilbert $V$. Bisa dibuktikan bahwa ruang $L^2$ dan ruang Sobolev adalah ruang Hilbert. Maka, konstruksi yang ada di sini adalah perumuman dari metode beda hingga. \par
		Misalkan $V$ adalah ruang Hilbert, $a(\cdot,\cdot):V\times V\to \mathbb{R}$ adalah fungsi yang linear terhadap kedua argumennya (disebut \emph{bilinear}), dan $l\in V'$. Kita akan tinjau masalah
		\begin{align}\label{fdm_cons1}
			u\in V, \, a(u,v) = l(v) \, \forall v\in V,
		\end{align}
		dengan asumsi $a(\cdot,\cdot)$ terbatas dan $V$-eliptik
		\begin{align}
			|a(u,v)|&\leq M\|u\|_V\|v\|_V \, \forall u,v\in V, \label{fdm_cons2} \\
			a(v,v) &\geq c_0\|v\|_V^2 \, \forall v\in V. \label{fdm_cons3}
		\end{align}
		untuk suatu $M,c_0>0$. Bisa dibuktikan bahwa masalah \eqref{fdm_cons1} memiliki solusi tunggal. Tapi, secara umum, sulit menyelesaikan masalah tersebut karena $V$ biasanya berdimensi tak hingga. 
	\end{frame}
	\begin{frame}{Metode Beda Hingga: Konstruksi}
		Dengan asumsi \eqref{fdm_cons2}, \eqref{fdm_cons3}, kita batasi masalah \eqref{fdm_cons1} ke subruang $V_n\subset V$ yang berdimensi $n$, yaitu
		\begin{align}\label{fdm_cons4}
			u_n\in V_n, \, a(u_n, v) = l(v), \, \forall v\in V_n.
		\end{align}
		Karena $V_n$ berdimensi hingga, kita bisa definisikan basis $\{\phi_i\}_{i=1}^n$ sehingga
		\begin{align*}
			u_n = \sum_{j=1}^{n} \xi_j\phi_j.
		\end{align*}
		Sekarang dengan menuliskan $v$ sebagai $\phi_i$ di \eqref{fdm_cons4} untuk setiap $i$, kita dapatkan SPL
		\begin{align}
			A\vec{\xi} = \vec{b}
		\end{align}
		dengan $A = [a(\phi_j,\phi_i)]$, $\vec{\xi} = [\xi_j]$, dan $\vec{b} = [l(\phi_i)]$. Maka \eqref{fdm_cons4} bisa diselesaikan dengan cara menyelesaikan SPL seperti di ALE.
	\end{frame}
	\begin{frame}{Metode Beda Hingga: Konstruksi}
		Tentunya solusi $u_n$ secara umum akan berbeda dari $u$. Agar solusi lebih akurat, bisa ditentukan solusi $u_N$ di $V_N\subset V$ sehingga $N>n$. Maka, untuk suatu barisan subruang $V_{n_1} \subset V_{n_2} \subset \cdots \subset V$, kita bisa hitung suatu barisan solusi $u_{n_i}\in V_{n_i}$. Prosedur ini disebut \emph{metode Galerkin}. \newline
		Sekarang kita akan tinjau masalah \eqref{fdm_cons1} di ruang Sobolev $H^2(a,b)$.
	\end{frame}
\end{document}
