\documentclass[xcolor={dvipsnames}]{beamer}
\usetheme{default}
\usepackage{amsmath, amsfonts, tikz, xcolor, biblatex}
\usepackage[cal = esstix]{mathalpha}
\usefonttheme{serif}

\addbibresource{referensi.bib}

\setbeamercolor{background canvas}{bg=blue!10!White}
\newcommand{\emp}[1]{\textcolor{Blue}{#1}}

\title{MatNum: Pengantar}
\author{Fadhlannafis K. K.}
\date{10122040}
\begin{document}
	\begin{frame}[plain]
		\maketitle
	\end{frame}
	\begin{frame}{Catatan}
		PPT ini saya buat untuk menunjang pembelajaran matematika numerik saya, setelah UTS saya di bawah setengahnya rata-rata (alias kuartil 4). \newline
		Pembahasan di PPT ini cenderung teoritik. Saya mengambil referensi dari PPT Bu Lena dan dari:
		\nocite{*}
		\printbibliography
	\end{frame}
	\begin{frame}{Apa Itu Numerik?}
		\emp{Matematika Numerik} adalah ilmu yang mempelajari prosedur untuk memecahkan masalah matematika menggunakan perhitungan yang sederhana. \newline
		\emp{Analisis Numerik} adalah matematika numerik, dengan mencakup pembuktian rumus-rumus yang diberikan. \newline
		\emp{Metode Numerik} adalah matematika numerik, dengan lebih menekankan pada penggunaan rumus.
	\end{frame}
	\begin{frame}{Tahapan}
		content...
	\end{frame}
	\begin{frame}{Algoritma}
		content...
	\end{frame}
	\begin{frame}{Galat}
		content...
	\end{frame}
	\begin{frame}{Angka Signifikan}
		content...
	\end{frame}
	\begin{frame}{Perlunya Analisis}
		Pada numerik, kita akan sering berhadapan dengan masalah \emp{konvergensi} dari metode yang kita pilih. \newline
		Nantinya, kita akan memperhatikan apakah barisan:
		\begin{enumerate}
			\item nilai hampiran akar,
			\item fungsi hampiran,
			\item vektor hampiran vektor eigen,
			\item dan lain sebagainya,
		\end{enumerate}
		itu konvergen ke hasil analitiknya. Untuk itu, diperlukan \emp{Analisis Fungsional}.
	\end{frame}
	\begin{frame}{Ruang Metrik}
		Misalkan $V$ adalah ruang vektor atas lapangan $F$. Kita sebut $V$ sebagai \emp{ruang metrik} jika ada fungsi $d:V\times V\to [0,\infty)$ sehingga untuk setiap $u,v,w\in V$:
		\begin{enumerate}
			\item $d(v,w) \geq 0$, dan $d(v,w) = 0$ jhj $v=w$,
			\item $d(v,w) = d(w,v)$,
			\item $d(v,w) \leq d(v,u) + d(u,w)$.
		\end{enumerate}
	\end{frame}
	\begin{frame}{}
		content...
	\end{frame}
\end{document}