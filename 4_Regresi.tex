\documentclass[xcolor={dvipsnames}, 9pt]{beamer}
\usetheme{default}
\usepackage{amsmath, amsfonts, tikz, xcolor, biblatex}
\usepackage[cal = esstix]{mathalpha}
\usefonttheme{serif}

\addbibresource{referensi.bib}

\setbeamercolor{background canvas}{bg=blue!10!White}
\renewcommand{\emph}[1]{\textcolor{Blue}{#1}}

\title{Judul}
\author{Fadhlannafis K. K.}
\date{10122040}
\begin{document}
	\begin{frame}[plain]
		\maketitle
	\end{frame}
	\begin{frame}{Daftar Isi}
		\tableofcontents
	\end{frame}
	\begin{frame}{Catatan}
		PPT ini saya buat untuk menunjang pembelajaran matematika numerik saya, setelah UTS saya di bawah setengahnya rata-rata (alias kuartil 4). \newline
		Pembahasan di PPT ini cenderung teoritik. Saya mengambil referensi dari PPT Bu Lena dan dari:
		\nocite{*}
		\printbibliography
	\end{frame}
	\section{Pengantar Regresi}
	\begin{frame}{Pengantar Regresi}
		Misalkan $f$ adalah fungsi kontinu di $[a,b]$
	\end{frame}
\end{document}
